\documentclass{article}

\usepackage{xcolor} % colorbox for the inline code command.
\usepackage{hyperref} % Hyperlinks and urls.
\usepackage{listings} % code boxes.
\usepackage{amsfonts} % \circledR.
\usepackage{graphicx} % Displaying pdf graphics.
\usepackage{multirow} % Used for specialized tables (multirow).
\usepackage{rotating} % Used to rotate figures sideways.
\usepackage{enumitem} % Compact lists.
\usepackage[mediumspace,mediumqspace,squaren,binary]{SIunits} % \milli\second

% Compact lists (see enumitem):
\setitemize{noitemsep,topsep=0pt,parsep=0pt,partopsep=0pt}

\newcommand{\masccite}[2][]{\cite{#2}}
\newcommand{\masccodeinline}[1]{\colorbox{black!5}{\lstinline[basicstyle=\ttfamily\color{black}]{#1}}}
\newread\file
\newcommand{\mascfirstline}[1]{\input{#1}\unskip}

\begin{document}

\title{Accelerating Graphics in the Simics Full-system Simulator}
\author{Eric Nilsson, Daniel Aarno, Erik Carstensen}
\maketitle

\section{A}
Virtual platforms provide benefits to developers in terms of a more rapid development cycle since development may begin before next-generation hardware is available.
However, there is a distinct lack of graphics virtualization in industry-grade virtual platforms, leading to performance issues that may reduce the benefits virtual platforms otherwise have over execution on actual hardware.

This paper concerns graphics acceleration by the means of paravirtualizing OpenGL ES 2.0 in the Wind River Simics full-system simulator.
The study illustrates the benefits and drawbacks of paravirtualized graphics acceleration and presents a performance analysis of strengths and weaknesses compared to software rasterization.
We propose a solution for paravirtualized graphics using magic instructions to share memory between target and host systems, and present an implementation utilizing this method.
Additionally, benchmarks are devised to stress key aspects in the solution, such as communication latency and computationally intensive applications.

\section{B}
In regards to previous work in the area, there is no indication - in academic writing - of existing paravirtualized graphics in a simulator with advanced capabilities like Simics, featuring deterministic execution, checkpointing and reverse execution.
Potential performance gains on such a platform are inherently unclear due to these features, and accordingly incentivize a performance evaluation.
Such functionality could simplify debugging, testing, and profiling of applications comprising some GPU-bound workload, leveraging the benefits of using Simics for software and systems development with applications requiring graphics acceleration, e.g. responsive UIs.

We present paravirtualization as a viable method to accelerate graphics in system simulators; this reduces frame times up to 34 times compared to that of software rasterization.
Furthermore, magic instructions are identified as the primary bottleneck of communication latency in the implementation.

\noindent\rule{8cm}{0.4pt}

Paravirtualized graphics acceleration is achieved through three overall components (see figure \ref{fig:overview}).
The target system libraries implement the EGL and OpenGL APIs and lures whatever application it is being linked to that it is, in fact, the expected platform libraries.
However, instead of communicating with the platform windowing system (in terms of EGL) and the graphics device (in terms of OpenGL), the target system libraries serialize the given command stream and forward it to the simulation host.
The host system libraries subsequently interprets the recieved byte stream, and performs the concerned OpenGL functions.
When the OpenGL invocation has been performed, any function results are returned to the target system.

Communications between target and host system libraries are performed using magic instructions.
Magic instructions are \masccodeinline{nop}-type instructions that, when executed on the simulated hardware in a virtual platform, invokes a callback-method in the simulation host~\masccite[p.~32]{publications:leupers:2010}.
During such an instruction, transmit a memory address in a 64 bit register -- the starting address of the serialized command stream.
Having successfully escaped the simulation context, we may assume the presence of a target virtual address in said CPU register and translate the given virtual address using the target system MMU.
This physical address can, in turn, be used to locate the given area in the simulated RAM image, thereby allowing direct access by the simulation host into the entire target memory page.

To evaluate the implementation, two benchmarks are devised.
These benchmarks stress suspected bottleneck in the implementation: one benchmark performs a large number of relatively insignificant OpenGL invocations, while the other has a computationally intensive workload.
As such, the benchmarks should be representative of typical scenarios induced by modern applications using OpenGL, such as responsive UIs.

During the experiments, the elapsed times of $1000$ frames is measured for each benchmark in software rasterized and paravirtualized Simics platforms, utilizing hardware-assisted virtualization so speed up simulation.
In order to gain some understanding on how well the given performance scales, we run three instances of each benchmark with different input.
Paravirtualized results are then compared with software rasterization.

\noindent\rule{8cm}{0.4pt}

Results accumulated from software rasterized and paravirtualized execution in Simics are presented in Tables~\ref{tab:keyvalsimics}~and~\ref{tab:keyvalpara}.

The Chess benchmark is designed to locate any bottlenecks related to the number of paravirtualized invocations, which is a predicted bottleneck.
Evidently, the prediction of a target-to-host communication latency issue has been confirmed, arguably identifying one weakness of graphics paravirtualization in the Simics full-system simulator.

In Simics, magic instructions incur a context switch cost when exiting the simulation and resuming execution on the host.
This affects the performance by forcing the simulation to no longer be executed natively, inhibiting the performance improvements granted by hardware-assisted virtualization.
It also entails Simics no longer being able to utilize just-in-time compilation to speed up execution, forcing the simulator to rely on regular code interpretation.
As such, in great numbers, magic instructions may greatly affect performance.
In order to establish what those overhead costs may be, further study into this matter is performed by measuring the elapsed time of escaping simulation $1000$ times using magic instructions.
From these findings, we may conclude that $1000$ consecutive magic instructions induce an average overhead of roughly $5$~\milli\second , minus profiling cost.
These findings indicate that magic instruction overhead could account for the majority of elapsed frame times when paravirtualized in Simics.

The Julia benchmark is incorporated into the experiment to establish how the paravirtualized solution performed under computational stress, which is where benefits induced by hardware acceleration should be made apparent.
Using this benchmark, we highlight weaknesses in Simics software rasterization, with frame times well above the two second mark; the corresponding maximum frame time in the paravirtualized Simics platform measuring up to to a mere $156$ \milli\second .
As visualized in Figure \ref{fig:histogramssimicsparajulia}, we showcase considerable performance improvements and -- in turn -- identify the capabilities of graphics paravirtualization in the Simics full-system simulator.

If hardware-assisted virtualization is not available, such as if the simulated platform is PowerPC, we may expect a major hit to performance.
For software rasterization, this impact accounts for well over two orders of magnitude increased frame time.
Meanwhile, performance impacts to paravirtualized Simics is not significant, sometimes as low as a third of the original frame time, except for the Chess benchmark where the frame time may increase with \textit{up to} one order of magnitude; still one order of magnitude less than the performance impact to software rasterization.
This can be expected, since the paravirtualized method entails that most work is performed on the simulation host.

Across the board, the paravirtualized solution suffer less performance impact, rendering the benchmarks at up to three orders of magnitude reduced frame times compared to software rasterization.
Accordingly, compared to execution with hardware-assisted virtualization, the effects of paravirtualization are increased by one order of magnitude.
This entails that workloads that are otherwise sub-optimal for paravirtualized graphics acceleration -- those performing many paravirtualized function invocations -- bring about performance improvements when utilizing paravirtualization.
Thus, the impact magic instructions have on performance is reduced when hardware-assisted virtualization is not available, likely because a magic instruction does not have to impose a costly context switch when exiting native execution.

We may conclude that some workloads (Chess) may attain decent simulation performance when software rasterized simply because of a fast simulator.
When native execution is not available, neither JIT compilation or interpretation can gain the same speeds as paravirtualization.

In Section \ref{sec:results}, we establish strengths and weaknesses of paravirtualized graphics in the Simics full-system simulator; most notably, the bottleneck introduced by magic instruction overhead.
As such, we confirm original suspicions through the use of the described benchmarks.
Thus, this study identifies the performance bottleneck inherent in great numbers of paravirtualized function invocations.

Furthermore, compiled results showcase great improvements for computationally intensive graphics, as demonstrated by the Julia fractal benchmark, compared to its software rasterized Simics counterpart.
As such, we have accelerated graphics by up to $34$ times, reducing frame time from that of $1415$~\milli\second\ to the real-time feasible count of 42~\milli\second ; a frame time of 42~\milli\second\ roughly correspond to 24 frames per second.
For platforms that cannot utilize hardware-assisted virtualization, paravirtualization is estimated to obtain an additional order of magnitude in frame time reduction, in comparison to software rasterization.
Such scenarios may include simulating other platforms than x86, such as PowerPC, or when utilizing certain advanced features in Simics -- e.g. breakpoints.

Accordingly, this experiment has identified the potential of using paravirtualization with shared memory for the means of accelerating graphics to that of real-time performance, testimonial to the results presented by Lagar-Cavilla et al. in their work on using paravirtualization to accelerate graphics~\masccite{inproceedings:lagarcavilla:2007}.
Additionally, results indicate performance improvements in terms of maximum frame times, inducing significantly improved standard deviation.
In line with stable frame rates being prerequisites for real-time applications, this further indicates, in coagency with reduced frame times, the feasibility of utilizing paravirtualized methodology for the purposes of accelerating graphics within virtual platforms.

To summarize: this paper presents a solution for graphics acceleration implemented in the Simics full-system simulator by the means of paravirtualization.
The end-result is a solution which may generate libraries imitating EGL and OpenGL libraries.
This solution may spy on application EGL utilization, without inhibiting said exchange, allowing unmodified OpenGL applications to be accelerated from within the simulation target.
The implementation communicates with low-latency magic instructions with no limit as to how much memory may be shared.
As such, throughout the paper, we have tackled and presented several issues pertaining to paravirtualized graphics acceleration.
For the purposes of performance testing, we have developed benchmarks with the distinct purpose of highlighting solution weaknesses and strengths.
We have presented an analysis of benchmarking results and concluded the benefits and drawbacks of paravirtualization as means to graphics acceleration in virtual platforms, backed by hard data stressing key points in the implementation.
Accordingly, the findings of this paper has contributed to our understanding of the difficulties facing paravirtualized graphics acceleration, and has established the feasibility of using paravirtualization to accelerate graphics in virtual platforms to that of real-time qualities.

To conclude: this paper has demonstrated performance improvements by accelerating graphics using paravirtualization.
Induced benefits are performance improvements of up to $34$ times -- or two orders of magnitude, paired with larger benefits in non-hardware-accelerated use-cases.
Magic instruction overhead is identified as the main performance bottleneck.
As such, a drawback of graphics paravirtualization is a weakness to large amounts of framework invocations.
Thus, this paper claims paravirtualization as a successful formula for system simulator graphics acceleration, and suggests utilizing high-level paravirtualization to accelerate graphics in virtual platforms.

\bibliographystyle{IEEEtran}
\bibliography{mascots}

\begin{figure}
\centering
\includegraphics[width=\linewidth]{img/yedoverview.pdf}
\caption{Overview of paravirtualized graphics in Simics.}
\label{fig:overview}
\end{figure}

\begin{figure}
\centering
\includegraphics[width=\linewidth]{img/yedvirtualmemory.pdf}
\caption{Simics Pipe overview. The target system libraries gives a virtual memory address to the host system libraries, which inquiries the target MMU in order to retrieve the byte stream from target physical memory.}
\label{fig:virtualmemory}
\end{figure}

\providecommand{\chesskeyone}{$60\times60$ tiles}
\providecommand{\chesskeytwo}{$84\times84$ tiles}
\providecommand{\chesskeythree}{$118\times118$ tiles}

\providecommand{\juliakeyone}{$225$ iterations}
\providecommand{\juliakeytwo}{$450$ iterations}
\providecommand{\juliakeythree}{$900$ iterations}

\begin{table*}
  \parbox{.5\textwidth}{
    \centering
    \tabcolsep=0.11cm
    \begin{tabular}{|c|c|c|c|c|c|}
      \hline
      \multirow{2}{*}{Benchmark} & \multirow{2}{*}{Input} & \multicolumn{4}{p{4cm}|}{\centering Elapsed time (\milli\second )} \\
      \cline{3-6} && \multicolumn{1}{c|}{Min} & \multicolumn{1}{c|}{Max} & \multicolumn{1}{c|}{Std} & \multicolumn{1}{c|}{Avg} \\ \hline
      \multirow{3}{*}{Chess} & \chesskeyone & \mascfirstline{simicschess60x60.dat.min} & \mascfirstline{simicschess60x60.dat.max}	& \mascfirstline{simicschess60x60.dat.std} & \mascfirstline{simicschess60x60.dat.avg} \\ %\cline{2-6}
      & \chesskeytwo & \mascfirstline{simicschess84x84.dat.min} & \mascfirstline{simicschess84x84.dat.max} & \mascfirstline{simicschess84x84.dat.std} & \mascfirstline{simicschess84x84.dat.avg} \\ %\cline{2-6}
      & \chesskeythree & \mascfirstline{simicschess118x118.dat.min} & \mascfirstline{simicschess118x118.dat.max} & \mascfirstline{simicschess118x118.dat.std} & \mascfirstline{simicschess118x118.dat.avg} \\ \hline
      \multirow{3}{*}{Julia} & \juliakeyone & \mascfirstline{simicsjulia225.dat.min} & \mascfirstline{simicsjulia225.dat.max} & \mascfirstline{simicsjulia225.dat.std} & \mascfirstline{simicsjulia225.dat.avg} \\ %\cline{2-6}
      & \juliakeytwo & \mascfirstline{simicsjulia450.dat.min} & \mascfirstline{simicsjulia450.dat.max} & \mascfirstline{simicsjulia450.dat.std} & \mascfirstline{simicsjulia450.dat.avg} \\ %\cline{2-6}
      & \juliakeythree & \mascfirstline{simicsjulia900.dat.min} & \mascfirstline{simicsjulia900.dat.max} & \mascfirstline{simicsjulia900.dat.std} & \mascfirstline{simicsjulia900.dat.avg} \\ \hline
    \end{tabular}
    \caption{Software rasterization results in Simics.}
    \label{tab:keyvalsimics}
  }
  \hfill
  \parbox{.5\textwidth}{
    \centering
    \begin{tabular}{|c|c|c|c|c|c|}
      \hline
      \multirow{2}{*}{Benchmark} & \multirow{2}{*}{Input} & \multicolumn{4}{p{4cm}|}{\centering Elapsed time (\milli\second )} \\
      \cline{3-6} && \multicolumn{1}{c|}{Min} & \multicolumn{1}{c|}{Max} & \multicolumn{1}{c|}{Std} & \multicolumn{1}{c|}{Avg} \\ \hline
      \multirow{3}{*}{Chess} & \chesskeyone & \mascfirstline{parachess60x60.dat.min} & \mascfirstline{parachess60x60.dat.max} & \mascfirstline{parachess60x60.dat.std} & \mascfirstline{parachess60x60.dat.avg} \\
      & \chesskeytwo & \mascfirstline{parachess84x84.dat.min} & \mascfirstline{parachess84x84.dat.max} & \mascfirstline{parachess84x84.dat.std} & \mascfirstline{parachess84x84.dat.avg} \\
      & \chesskeythree & \mascfirstline{parachess118x118.dat.min} & \mascfirstline{parachess118x118.dat.max} & \mascfirstline{parachess118x118.dat.std} & \mascfirstline{parachess118x118.dat.avg} \\ \hline
      \multirow{3}{*}{Julia} & \juliakeyone & \mascfirstline{parajulia225.dat.min} & \mascfirstline{parajulia225.dat.max}	& \mascfirstline{parajulia225.dat.std} & \mascfirstline{parajulia225.dat.avg} \\
      & \juliakeytwo & \mascfirstline{parajulia450.dat.min} & \mascfirstline{parajulia450.dat.max} & \mascfirstline{parajulia450.dat.std} & \mascfirstline{parajulia450.dat.avg} \\
      & \juliakeythree & \mascfirstline{parajulia900.dat.min} & \mascfirstline{parajulia900.dat.max} & \mascfirstline{parajulia900.dat.std} & \mascfirstline{parajulia900.dat.avg} \\ \hline
    \end{tabular}
    \caption{Paravirtualization results in Simics.}
    \label{tab:keyvalpara}
  }
\end{table*}

\end{document}
